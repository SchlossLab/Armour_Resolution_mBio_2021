% Options for packages loaded elsewhere
\PassOptionsToPackage{unicode}{hyperref}
\PassOptionsToPackage{hyphens}{url}
%
\documentclass[
]{article}
\usepackage{amsmath,amssymb}
\usepackage{lmodern}
\usepackage{ifxetex,ifluatex}
\ifnum 0\ifxetex 1\fi\ifluatex 1\fi=0 % if pdftex
  \usepackage[T1]{fontenc}
  \usepackage[utf8]{inputenc}
  \usepackage{textcomp} % provide euro and other symbols
\else % if luatex or xetex
  \usepackage{unicode-math}
  \defaultfontfeatures{Scale=MatchLowercase}
  \defaultfontfeatures[\rmfamily]{Ligatures=TeX,Scale=1}
\fi
% Use upquote if available, for straight quotes in verbatim environments
\IfFileExists{upquote.sty}{\usepackage{upquote}}{}
\IfFileExists{microtype.sty}{% use microtype if available
  \usepackage[]{microtype}
  \UseMicrotypeSet[protrusion]{basicmath} % disable protrusion for tt fonts
}{}
\makeatletter
\@ifundefined{KOMAClassName}{% if non-KOMA class
  \IfFileExists{parskip.sty}{%
    \usepackage{parskip}
  }{% else
    \setlength{\parindent}{0pt}
    \setlength{\parskip}{6pt plus 2pt minus 1pt}}
}{% if KOMA class
  \KOMAoptions{parskip=half}}
\makeatother
\usepackage{xcolor}
\IfFileExists{xurl.sty}{\usepackage{xurl}}{} % add URL line breaks if available
\IfFileExists{bookmark.sty}{\usepackage{bookmark}}{\usepackage{hyperref}}
\hypersetup{
  hidelinks,
  pdfcreator={LaTeX via pandoc}}
\urlstyle{same} % disable monospaced font for URLs
\usepackage[margin=1.0in]{geometry}
\usepackage{longtable,booktabs,array}
\usepackage{calc} % for calculating minipage widths
% Correct order of tables after \paragraph or \subparagraph
\usepackage{etoolbox}
\makeatletter
\patchcmd\longtable{\par}{\if@noskipsec\mbox{}\fi\par}{}{}
\makeatother
% Allow footnotes in longtable head/foot
\IfFileExists{footnotehyper.sty}{\usepackage{footnotehyper}}{\usepackage{footnote}}
\makesavenoteenv{longtable}
\usepackage{graphicx}
\makeatletter
\def\maxwidth{\ifdim\Gin@nat@width>\linewidth\linewidth\else\Gin@nat@width\fi}
\def\maxheight{\ifdim\Gin@nat@height>\textheight\textheight\else\Gin@nat@height\fi}
\makeatother
% Scale images if necessary, so that they will not overflow the page
% margins by default, and it is still possible to overwrite the defaults
% using explicit options in \includegraphics[width, height, ...]{}
\setkeys{Gin}{width=\maxwidth,height=\maxheight,keepaspectratio}
% Set default figure placement to htbp
\makeatletter
\def\fps@figure{htbp}
\makeatother
\setlength{\emergencystretch}{3em} % prevent overfull lines
\providecommand{\tightlist}{%
  \setlength{\itemsep}{0pt}\setlength{\parskip}{0pt}}
\setcounter{secnumdepth}{-\maxdimen} % remove section numbering
\usepackage{helvet}
\renewcommand*\familydefault{\sfdefault}
\usepackage{setspace}
\doublespacing
\usepackage[left]{lineno}
\linenumbers
\ifluatex
  \usepackage{selnolig}  % disable illegal ligatures
\fi

\author{}
\date{\vspace{-2.5em}}

\begin{document}

\hypertarget{optimal-taxonomic-resolution-for-microbiome-based-classification-of-colorectal-cancer}{%
\section{Optimal Taxonomic Resolution for Microbiome-Based
Classification of Colorectal
Cancer}\label{optimal-taxonomic-resolution-for-microbiome-based-classification-of-colorectal-cancer}}

\vspace{20mm}

\textbf{Other title options:}\\
ASV level resolution is unnecessary for prediction of SRNs from
microbiome data\\
OTUs are the Optimal Taxonomic Level for Microbiome-Based Classification
of Colorectal Cancer

\vspace{10mm}

Courtney R. Armour, (ande? begum?), Patrick D. Schloss \({^\dagger}\)

\vspace{20mm}

\({\dagger}\) To whom correspondence should be addressed:

\href{mailto:pschloss@umich.edu}{pschloss@umich.edu}

Department of Microbiology

University of Michigan

Ann Arbor, MI 48109

\vspace{20mm}

\textbf{observation format - max 1200 words, 2 figures, 25 ref}

\newpage

\hypertarget{abstract-max-250-words}{%
\subsection{Abstract (max 250 words)}\label{abstract-max-250-words}}

\hypertarget{importance-max-150-words}{%
\subsection{Importance (max 150 words)}\label{importance-max-150-words}}

\newpage

\hypertarget{introduction}{%
\subsection{Introduction}\label{introduction}}

\begin{itemize}
\tightlist
\item
  Colorectal cancer (CRC) is one of the most common cancers in men and
  women and a leading cause of cancer related death in the United States
  \{\}.
\item
  Early detection and treatment are essential to increase survival
  rates, but for a variety of reasons including the invasiveness and
  high cost of screening (i.e.~colonoscopy), many people do not comply
  with recommended screening guidelines \{\}.
\item
  Less invasive and more affordable screening methods (e.g.~fecal
  immunochemical test) are available, however these tests are less
  sensitive than colonoscopies, especially for detecting early stage
  adenomas.
\item
  A growing body of research points to the gut microbiome as having a
  role in tumor development and progression \{\}. For example, studies
  find that Fusobacterium nucleatum and enterotoxigenic Bacteriodes
  fragilis tend to be enriched in the gut microbiome of subjects with
  CRC relative to healthy controls \{\}, while potentially protective
  bacteria such as members of Lachnospiraceae tend to be depleted \{\}.
\item
  The gut microbiome shows promise as diagnostic \{\}
\item
  Efforts to realize the diagnostic potential of the gut microbiome in
  detecting screen relevant neoplasias (SRNs) have focused on machine
  learning (ML) methods using abundances from operational taxonomic unit
  (OTU) classifications based on the amplicon sequencing of the v4
  region of the 16S rRNA gene \{\}. However, whether this is the optimal
  taxonomic resolution for classifying SRNs from microbiome data is
  unknown.

  \begin{itemize}
  \tightlist
  \item
    Additionally, recent work has pushed for the use of amplicon
    sequence variants (ASVs) to replace OTUs for marker-gene analysis
    because of the improved resolution with ASVs. However, whether the
    additional resolution provided by ASVs is useful for ML
    classification is unclear.

    \begin{itemize}
    \tightlist
    \item
      Since ML classification relies on consistent differences between
      groups, its possible that the resolution at the ASV level is too
      individualized to accurately differentiate groups.
    \end{itemize}
  \end{itemize}
\item
  Topçuoğlu \emph{et al} (mBio 2020) recently demonstrated effective
  application of machine learning (ML) to microbiome based
  classification problems and developed a framework for applying ML
  practices in a more reproducible way (mikropml).
\item
  This analysis utilizes the reproducible framework developed by
  Topçuoğlu \emph{et al} to quantify which ML method and taxonomic level
  produce the best performing classifier for detecting SRNs from
  microbiome data.
\end{itemize}

\hypertarget{results}{%
\subsection{Results}\label{results}}

\begin{itemize}
\tightlist
\item
  Across the five ML methods tested, model performance tended to
  increase with taxonomic level usually peaking around genus/OTU level
  before dropping off slightly with ASVs.
  \protect\hyperlink{model-performance-across-taxonomy}{(Figure)}
\item
  Random forest was consistently the top performer at most taxonomic
  levels.

  \begin{itemize}
  \tightlist
  \item
    RF might be more appropriate for microbiome analysis since its more
    suitable for zero inflated data
  \item
    TODO: dig into literature on this
  \item
    TODO: address lower L2 logistic performance that prior results?
  \end{itemize}
\item
  Within the RF model, the highest AUCs were observed for family (median
  AUC: 0.687), genus(median AUC: 0.686), and OTU (median AUC: 0.698)
  level data with no significant difference between the three.
  \protect\hyperlink{random-forest-model-performance-with-significance}{(Figure)}

  \begin{itemize}
  \tightlist
  \item
    ASV (median AUC: 0.676) performed significantly lower than OTU and
    genus (p \textless{} 0.05) but not family level P = 0.06).
  \end{itemize}
\item
  One hypothesis for the observation that model performance increases
  from phylum to OTU level then drops slightly at ASV level is that at
  higher taxonomic levels (e.g.~phylum) there are too few taxa and too
  much overlap to reliably differentiate between cases and controls.

  \begin{itemize}
  \tightlist
  \item
    As you reach genus/OTU level data there is enough data and variation
    but at the ASV level, the data is too specific to individuals and
    doesn't overlap enough.
  \item
    Examination of the prevalence of taxa in samples at each level
    supports this idea. A majority of taxa are present in greater than
    75\% of samples at the phylum (67\% of taxa) and class (63\% of
    taxa) levels. The opposite is observed at the OTU and ASV level
    where 60\% and 53\% of taxa respectively are only present in less
    than 25\% of samples.
    \protect\hyperlink{prevalence-of-taxa-in-samples}{(Figure)}
  \end{itemize}
\item
  Of note, the ML pipeline includes a pre-processing step that occurs
  prior to training and classifying the ML models (methods). As part of
  this step, features are removed that wont provide useful information
  to build the model. For example, strongly correlated features provide
  the same information to build the model and thus can be collapsed.
  Additionally, features with zero or near-zero variance will not help
  the model differentiate groups and thus can be removed.

  \begin{itemize}
  \tightlist
  \item
    Interestingly, despite starting with 104106 features at the ASV
    level, only 478 (0.5\%) remained after pre-processing. At the OTU
    level, 20079 of the 705 features (3.5\%) remained after
    preprocessing. \protect\hyperlink{summary-of-features}{(Table)}
  \item
    TODO: check why ASVs were removed in preprocessing (nzv?)
  \item
    (depending on above can discuss why features were removed -
    e.g.~only in one or a few samples)
  \end{itemize}
\item
  OTU level data provides an idea balance of overlap and distinction to
  classify samples.
\item
  Overall, the fine resolution of ASVs is unnecessary for ML
  classification with microbiome data since the vast majority of the
  ASVs are removed during preprocessing and the model performance is
  diminshed compared to other taxonomic levels.
\end{itemize}

other items that could be addressed:

\begin{itemize}
\tightlist
\item
  what are important taxa, patterns along levels (e.g fuso always in top
  )
\item
  abundance of important taxa: many of the top important features are
  low abund
  \protect\hyperlink{relative-abundance-of-important-features}{(Figure)}
\end{itemize}

\hypertarget{conclusions}{%
\subsection{Conclusions}\label{conclusions}}

\hypertarget{materials-and-methods}{%
\subsection{Materials and Methods}\label{materials-and-methods}}

\begin{itemize}
\tightlist
\item
  16S rRNA data from 490 subjects \{baxter\}

  \begin{itemize}
  \tightlist
  \item
    261 controls
  \item
    229 cases
  \end{itemize}
\item
  processed with mothur v1.44.3

  \begin{itemize}
  \tightlist
  \item
    SILVA v132 reference
  \end{itemize}
\item
  ML with mikropml package

  \begin{itemize}
  \tightlist
  \item
    preprocessing -normalize values

    \begin{itemize}
    \tightlist
    \item
      what is removed and why

      \begin{itemize}
      \tightlist
      \item
        near zero variance \& zero variance removed
      \item
        correlated collapsed
      \end{itemize}
    \end{itemize}
  \item
    hp tuning
  \item
    n seeds
  \item
    for more detail see mikropml package\{\}
  \end{itemize}
\item
  pvalues calculated as previously described \{begum\}
\item
  prevalence = number of samples with non-zero abundance / total number
  of samples
\end{itemize}

\hypertarget{acknowledgements}{%
\subsection{Acknowledgements}\label{acknowledgements}}

\hypertarget{figures}{%
\subsection{Figures}\label{figures}}

\hypertarget{model-performance-across-taxonomy}{%
\subsubsection{Model Performance across
Taxonomy}\label{model-performance-across-taxonomy}}

\includegraphics{../exploratory/figures/all_model_level.png}

\hypertarget{random-forest-model-performance-with-significance}{%
\subsubsection{Random Forest Model Performance with
Significance}\label{random-forest-model-performance-with-significance}}

\includegraphics[width=0.75\textwidth,height=\textheight]{../exploratory/figures/rf_with_stats.png}

\hypertarget{prevalence-of-taxa-in-samples}{%
\subsubsection{Prevalence of Taxa in
Samples}\label{prevalence-of-taxa-in-samples}}

\includegraphics[width=\textwidth,height=0.5\textheight]{../exploratory/figures/bin_prevalence.png}

\hypertarget{summary-of-features}{%
\subsubsection{Summary of Features}\label{summary-of-features}}

\begin{longtable}[]{@{}lrrrr@{}}
\toprule
level & n\_samples & n\_features & n\_features\_preproc &
pct\_kept \\ \addlinespace
\midrule
\endhead
phylum & 490 & 19 & 9 & 47.4 \\ \addlinespace
class & 490 & 36 & 19 & 52.8 \\ \addlinespace
order & 490 & 65 & 28 & 43.1 \\ \addlinespace
family & 490 & 124 & 54 & 43.5 \\ \addlinespace
genus & 490 & 316 & 115 & 36.4 \\ \addlinespace
otu & 490 & 20079 & 705 & 3.5 \\ \addlinespace
asv & 490 & 104106 & 478 & 0.5 \\ \addlinespace
\bottomrule
\end{longtable}

\hypertarget{relative-abundance-of-important-features}{%
\subsubsection{Relative Abundance of Important
Features}\label{relative-abundance-of-important-features}}

\includegraphics[width=\textwidth,height=0.95\textheight]{../exploratory/figures/relabund_top10.png}

\end{document}
