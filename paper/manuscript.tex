% Options for packages loaded elsewhere
\PassOptionsToPackage{unicode}{hyperref}
\PassOptionsToPackage{hyphens}{url}
%
\documentclass[
]{article}
\usepackage{amsmath,amssymb}
\usepackage{lmodern}
\usepackage{ifxetex,ifluatex}
\ifnum 0\ifxetex 1\fi\ifluatex 1\fi=0 % if pdftex
  \usepackage[T1]{fontenc}
  \usepackage[utf8]{inputenc}
  \usepackage{textcomp} % provide euro and other symbols
\else % if luatex or xetex
  \usepackage{unicode-math}
  \defaultfontfeatures{Scale=MatchLowercase}
  \defaultfontfeatures[\rmfamily]{Ligatures=TeX,Scale=1}
\fi
% Use upquote if available, for straight quotes in verbatim environments
\IfFileExists{upquote.sty}{\usepackage{upquote}}{}
\IfFileExists{microtype.sty}{% use microtype if available
  \usepackage[]{microtype}
  \UseMicrotypeSet[protrusion]{basicmath} % disable protrusion for tt fonts
}{}
\makeatletter
\@ifundefined{KOMAClassName}{% if non-KOMA class
  \IfFileExists{parskip.sty}{%
    \usepackage{parskip}
  }{% else
    \setlength{\parindent}{0pt}
    \setlength{\parskip}{6pt plus 2pt minus 1pt}}
}{% if KOMA class
  \KOMAoptions{parskip=half}}
\makeatother
\usepackage{xcolor}
\IfFileExists{xurl.sty}{\usepackage{xurl}}{} % add URL line breaks if available
\IfFileExists{bookmark.sty}{\usepackage{bookmark}}{\usepackage{hyperref}}
\hypersetup{
  hidelinks,
  pdfcreator={LaTeX via pandoc}}
\urlstyle{same} % disable monospaced font for URLs
\usepackage[margin=1.0in]{geometry}
\usepackage{graphicx}
\makeatletter
\def\maxwidth{\ifdim\Gin@nat@width>\linewidth\linewidth\else\Gin@nat@width\fi}
\def\maxheight{\ifdim\Gin@nat@height>\textheight\textheight\else\Gin@nat@height\fi}
\makeatother
% Scale images if necessary, so that they will not overflow the page
% margins by default, and it is still possible to overwrite the defaults
% using explicit options in \includegraphics[width, height, ...]{}
\setkeys{Gin}{width=\maxwidth,height=\maxheight,keepaspectratio}
% Set default figure placement to htbp
\makeatletter
\def\fps@figure{htbp}
\makeatother
\setlength{\emergencystretch}{3em} % prevent overfull lines
\providecommand{\tightlist}{%
  \setlength{\itemsep}{0pt}\setlength{\parskip}{0pt}}
\setcounter{secnumdepth}{-\maxdimen} % remove section numbering
\usepackage{helvet}
\renewcommand*\familydefault{\sfdefault}
\usepackage{setspace}
\doublespacing
\usepackage[left]{lineno}
\linenumbers
\ifluatex
  \usepackage{selnolig}  % disable illegal ligatures
\fi

\author{}
\date{\vspace{-2.5em}}

\begin{document}

\hypertarget{title-goes-here}{%
\section{TITLE GOES HERE}\label{title-goes-here}}

\vspace{20mm}

\textbf{Running title:} Optimal resolution

\vspace{10mm}

Courtney R. Armour, (ande? begum? etc?), Patrick D. Schloss
\({^\dagger}\)

\vspace{20mm}

\({\dagger}\) To whom correspondence should be addressed:

\href{mailto:pschloss@umich.edu}{pschloss@umich.edu}

Department of Microbiology

University of Michigan

Ann Arbor, MI 48109

\vspace{20mm}

\textbf{(observation format - max 1200 words, 2 figures, 25 ref)}

\newpage

\hypertarget{abstract-max-250-words}{%
\subsection{Abstract (max 250 words)}\label{abstract-max-250-words}}

\hypertarget{importance-max-150-words}{%
\subsection{Importance (max 150 words)}\label{importance-max-150-words}}

\newpage

\hypertarget{introduction}{%
\subsection{Introduction}\label{introduction}}

\begin{itemize}
\tightlist
\item
  CRC is one of the most common cancers and a leading cause of cancer
  related death
\item
  There is evidence that the microbiome has a role in CRC
  development/progression and could be useful for biomarker detection
  and diagnostics.
\item
  Begum et al (mBio 2020) recently demonstrated effective application of
  machine learning (ML) to microbiome based classification problems and
  developed a framework for applying ML practices in a more reproducible
  way (mikropml).
\item
  A common question when applying ML methods to microbiome data is which
  method and taxonomic level should be use.
\item
  This analysis utilizes the reproducible framework developed by Begum
  et al to quantify which ML method and taxonomic level produce the best
  performing classifier for CRC data.
\end{itemize}

\newpage

\hypertarget{results}{%
\subsection{Results}\label{results}}

\begin{itemize}
\tightlist
\item
  Of the five ML methods tested, Random forest was consistently the top
  performer (supplemental figure of all models?) at most taxonomic
  levels.

  \begin{itemize}
  \tightlist
  \item
    RF might be more appropriate anyways since its more suitable for
    zero inflated data? (need to look into literature)
  \end{itemize}
\item
  Within the RF model, the highest AUCs were observed for family, genus,
  and otu level data with no significant difference between the three.
  (Figure 1)
\end{itemize}

\newpage

\hypertarget{conclusion}{%
\subsection{Conclusion}\label{conclusion}}

\newpage

\hypertarget{materials-and-methods}{%
\subsection{Materials and Methods}\label{materials-and-methods}}

\begin{itemize}
\tightlist
\item
  data from prior study \{baxter\}
\item
  mikropml package
\item
  pvalues as previously described \{begum\}
\end{itemize}

\newpage

\hypertarget{acknowledgements}{%
\subsection{Acknowledgements}\label{acknowledgements}}

\newpage

\hypertarget{figures}{%
\subsection{Figures}\label{figures}}

\end{document}
