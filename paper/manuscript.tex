\documentclass[]{article}
\usepackage[T1]{fontenc}
\usepackage{lmodern}
\usepackage{amssymb,amsmath}
\usepackage{ifxetex,ifluatex}
\usepackage{fixltx2e} % provides \textsubscript
% use upquote if available, for straight quotes in verbatim environments
\IfFileExists{upquote.sty}{\usepackage{upquote}}{}
\ifnum 0\ifxetex 1\fi\ifluatex 1\fi=0 % if pdftex
  \usepackage[utf8]{inputenc}
\else % if luatex or xelatex
  \ifxetex
    \usepackage{mathspec}
    \usepackage{xltxtra,xunicode}
  \else
    \usepackage{fontspec}
  \fi
  \defaultfontfeatures{Mapping=tex-text,Scale=MatchLowercase}
  \newcommand{\euro}{€}
\fi
% use microtype if available
\IfFileExists{microtype.sty}{\usepackage{microtype}}{}
\usepackage[margin=1.0in]{geometry}
\usepackage{graphicx}
% Redefine \includegraphics so that, unless explicit options are
% given, the image width will not exceed the width of the page.
% Images get their normal width if they fit onto the page, but
% are scaled down if they would overflow the margins.
\makeatletter
\def\ScaleIfNeeded{%
  \ifdim\Gin@nat@width>\linewidth
    \linewidth
  \else
    \Gin@nat@width
  \fi
}
\makeatother
\let\Oldincludegraphics\includegraphics
{%
 \catcode`\@=11\relax%
 \gdef\includegraphics{\@ifnextchar[{\Oldincludegraphics}{\Oldincludegraphics[width=\ScaleIfNeeded]}}%
}%
\ifxetex
  \usepackage[setpagesize=false, % page size defined by xetex
              unicode=false, % unicode breaks when used with xetex
              xetex]{hyperref}
\else
  \usepackage[unicode=true]{hyperref}
\fi
\hypersetup{breaklinks=true,
            bookmarks=true,
            pdfauthor={},
            pdftitle={},
            colorlinks=true,
            citecolor=blue,
            urlcolor=blue,
            linkcolor=magenta,
            pdfborder={0 0 0}}
\urlstyle{same}  % don't use monospace font for urls
\setlength{\parindent}{0pt}
\setlength{\parskip}{6pt plus 2pt minus 1pt}
\setlength{\emergencystretch}{3em}  % prevent overfull lines
\setcounter{secnumdepth}{0}

\author{}
\date{\vspace{-2.5em}}
\usepackage{helvet}
\renewcommand*\familydefault{\sfdefault}
\usepackage{setspace}
\doublespacing
\usepackage[left]{lineno}
\linenumbers

\begin{document}

\section{Optimal Resolution for Microbiome-Based Classification of
Colorectal
Cancer}\label{optimal-resolution-for-microbiome-based-classification-of-colorectal-cancer}

\vspace{20mm}

\textbf{Running title:} Optimal Resolution

\vspace{10mm}

Courtney R. Armour, (ande? begum?), Patrick D. Schloss ${^\dagger}$

\vspace{20mm}

${\dagger}$ To whom correspondence should be addressed:

\href{mailto:pschloss@umich.edu}{pschloss@umich.edu}

Department of Microbiology

University of Michigan

Ann Arbor, MI 48109

\vspace{20mm}

\textbf{observation format - max 1200 words, 2 figures, 25 ref}

\newpage

\subsection{Abstract (max 250 words)}\label{abstract-max-250-words}

\subsection{Importance (max 150 words)}\label{importance-max-150-words}

\newpage

\subsection{Introduction}\label{introduction}

\begin{itemize}
\itemsep1pt\parskip0pt\parsep0pt
\item
  CRC is one of the most common cancers and a leading cause of cancer
  related death
\item
  There is evidence that the microbiome has a role in CRC
  development/progression and could be useful for biomarker detection
  and diagnostics.
\item
  Begum et al (mBio 2020) recently demonstrated effective application of
  machine learning (ML) to microbiome based classification problems and
  developed a framework for applying ML practices in a more reproducible
  way (mikropml).
\item
  A common question when applying ML methods to microbiome data is which
  method and taxonomic level is optimal.
\item
  This analysis utilizes the reproducible framework developed by Begum
  et al to quantify which ML method and taxonomic level produce the best
  performing classifier for detecting SRNs from microbiome data.
\end{itemize}

\newpage

\subsection{Results}\label{results}

\begin{itemize}
\itemsep1pt\parskip0pt\parsep0pt
\item
  Across the five ML methods tested, model performance tends to increase
  with taxonomic level usually peaking around genus/otu level and
  dropping off slightly with ASVs.
\item
  Random forest was consistently the top performer at most taxonomic
  levels.
\item
  RF might be more appropriate for microbiome analysis since its more
  suitable for zero inflated data (need to look into literature)
\item
  Within the RF model, the highest AUCs were observed for family, genus,
  and otu level data with no significant difference between the three
  (Figure 1). ASV performed significantly lower than OTU but not family
  and genus levels.
\end{itemize}

\newpage

\subsection{Conclusion}\label{conclusion}

\newpage

\subsection{Materials and Methods}\label{materials-and-methods}

\begin{itemize}
\itemsep1pt\parskip0pt\parsep0pt
\item
  data from prior study \{baxter\}
\item
  mikropml package
\item
  preprocessing
\item
  hp tuning
\item
  n seeds
\item
  pvalues as previously described \{begum\}
\end{itemize}

\newpage

\subsection{Acknowledgements}\label{acknowledgements}

\newpage

\subsection{Figures}\label{figures}

\end{document}
